\section{Introduction}

% This article proposes to obtain a statistical model of the daily peak electricity load  of a household located in Austin-TX,USA. The data was gathered from the Pecan Street Project Database of the University of Texas at Austin. They account for minute-to-minute electricity use during the year of 2015 of the house (a total of 525600 observations)
% The programming language Python was used to create an algorithm to find the maximum minute-load of a day, for all the 365 days of the year. This data then served as input for for the statistical modeling, which was performed in the programming language R. 
% The data was separated between training data (first 10 months and test data (last two months). The training data was used to identify the model and the test data for the accuracy of the forecast. The next section will explain the methodology and then apply it to the case study. 

This report proposes a statistical model to predict whether a patient has symptoms of heart disease. The cleaned data was given prior to the analysis, which contains a mixture of categorical and numerical features determining the patient's demographics and clinical measures. Experiments are conducted on three different types of models: $k$-Nearest Neighbours ($k$-NN), Decision Trees and Logistic Regression. Models are evaluated based on sensitivity (true positive rate, or TPR) with \( 5 \)-fold validation, from which the best model from each of its kind is selected and compared based on ROC-AUC \textit{(Receiver Operating Characteristic - Area Under the Curve)} and Precision-Recall tradeoffs.